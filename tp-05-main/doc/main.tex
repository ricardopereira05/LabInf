<<<<<<< HEAD
\documentclass[a4paper,12pt]{report}

% Pacotes necessários
\usepackage[utf8]{inputenc}
\usepackage[portuguese]{babel}
\usepackage{float}
\usepackage{amsmath}
\usepackage{listings}
\usepackage{geometry}
\usepackage{graphicx}
\geometry{a4paper, margin=2.5cm}

% Configuração para listagens de código
\lstset{
    language=C,
    basicstyle=\ttfamily\small,
    numbers=left,
    stepnumber=1,
    numbersep=5pt,
    showspaces=false,
    showstringspaces=false,
    showtabs=false,
    frame=single,
    breaklines=true,
    breakatwhitespace=true,
    tabsize=4
}

% Início do documento
\begin{document}

% Capa
\begin{titlepage}
    \centering
    {\Large IPCA}\\[1.5cm]
    {\Large Relatório de Projeto de PI e Laboratórios de Informárica}\\[2cm]
    {\Huge \textbf{Relatório}}\\[2cm]
    {\large Autores: David Faria 31517, João Pereira 31505, Rodrigo Pinheiro 31502 } {\large Docente: Oscár Ribeiro}\\[1.5cm]
    {\large \today}\\
\end{titlepage}

\tableofcontents
\newpage

% Introdução
\chapter{Introdução}
O presente relatório foi elaborado no âmbito da Unidade Curricular de Laboratórios de Informática do curso de Licenciatura em Engenharia de Sistemas Informáticos, durante o ano letivo de 2024/2025. Este trabalho tem como objetivo consolidar competências práticas relacionadas com o desenvolvimento colaborativo de projetos utilizando a linguagem de programação C, adotando o paradigma de programação imperativa.
Ao longo do desenvolvimento do projeto, são abordadas boas práticas de programação, incluindo a modularização de código, o uso de ferramentas de automação como o Makefile e a documentação utilizando ferramentas como o Doxygen. Adicionalmente, promove-se a utilização de sistemas de gestão de versões, com ênfase na colaboração em grupo por meio da plataforma GitHub.
Outro aspeto central do trabalho é a utilização avançada da linha de comandos em ambientes Linux, promovendo uma interação eficiente com o sistema operativo para tarefas de desenvolvimento, compilação e execução do programa.
Este relatório documenta todas as etapas do trabalho como, por exemplo: a definição do problema e sua análise, até a implementação, resultados obtidos e considerações finais. 

\begin{itemize}
    \item Definição do problema e sua análise;
    \item Implementação de funções;
    \item Resultados obtidos;
    \item Considerações finais;
\end{itemize}

% Capítulo de Desenvolvimento
\chapter{Desenvolvimento}

\section{Conceito do programa}
O programa desenvolvido apoia a gestão do Espaço Social de uma Instituição Pública, responsável por servir refeições aos seus funcionários. Usando a linguagem C, o sistema permite gerir dados dos funcionários, ementas semanais e escolhas de refeições, garantindo assim a sua eficiência.

\textbf{Principais funcionalidades:}

\begin{itemize}
    \item Carregamento de Dados: Leitura de ficheiros com informações dos funcionários, ementas e escolhas de refeições.
    \item Gestão e Visualização: Listagem de refeições diárias, relatórios semanais para Recursos Humanos, e análise de calorias consumidas.
    \item Análise e Relatórios: Cálculo de médias de calorias e geração de tabelas detalhadas por funcionário.
    \item Flexibilidade: Suporte a ficheiros de texto e binários, com interface de linha de comandos para personalização.
\end{itemize}
O código é documentado com Doxygen, com um Makefile para compilação automatizada, e promove boas práticas de programação e trabalho em equipa.


\begin{description}
    \item Funcoes .h: Declaram funções e estruturas de dados, o que garante o armazenamento dos dados necessários.
    \item Funcoes .c: Implementam e desenvolve as funções declaradas anteriormente.  
    \item Makefile: Automatiza o processo de compilação, especificando dependências e otimizando a geração do executável
    \item Main.c: Coordena a execução do programa - processa argumentos da linha de comandos, chama funções e interage com o utilizador.
    \item ementas.txt: Contém as ementas semanais, organizadas por dia, com pratos de peixe, carne, dieta e vegetariano, incluindo as calorias de cada prato.
Exemplo:
Segunda;18.11.2024;peixe grelhado;180;bife de vaca;330;frango grelhado;150;lasanha de vegetais;200;
    \item escolhas.txt: Registra as escolhas dos funcionários, indicando o dia da semana, número de funcionário e o tipo de prato escolhido. Exemplo: Segunda;1;Peixe
    \item funcionarios.txt: Armazena os dados dos funcionários, como número, nome, NIF e telefone. Exemplo: 1;Paulo Silva;179204181;963358792
\end{description}

\subsection{Exemplos de funções implementadas}
Segue abaixo exemplos de funções implementadas no código:

\begin{lstlisting}[caption={carregarFuncionarios}, label={lst:main}]
/**
@brief  Função para carregar os dados dos funcionários a partir de um ficheiro
@param nomeFicheiro   Nome do ficheiro
@param funcionario  Array de funcionários
@return int  Retorna o total de funcionários carregados*/
int carregarFuncionarios(char* nomeFicheiro, Funcionario funcionario[]) {
    FILE *ficheiro = fopen(nomeFicheiro, "r");
    if (ficheiro == NULL) {
        printf("Erro ao abrir o ficheiro: %s\n", nomeFicheiro);
        return 0;
    }

    int totalFuncionarios = 0;
    while (totalFuncionarios < MAX_FUNCIONARIOS) {

        // Lê os dados do ficheiro e atribui aos campos da estrutura
        if (fscanf(ficheiro, "%d;%99[^;];%14[^;];%14[^\n]\n",
                   &funcionario[totalFuncionarios].numero,
                   funcionario[totalFuncionarios].nome,
                   funcionario[totalFuncionarios].nif,
                   funcionario[totalFuncionarios].telefone) == 4) {
            totalFuncionarios++;
        } else {
            break; // Encerra a leitura caso ocorra erro ou final do arquivo
        }
    }

    fclose(ficheiro);

    printf("Carregamento dos dados dos funcionários foi realizado com sucesso!\n");
    printf("Total de funcionários carregados: %d\n", totalFuncionarios);
    return totalFuncionarios;
    } 
\end{lstlisting}

\begin{lstlisting}[caption={carregarEmenta}, label={lst:main}]
/**
 
@brief  Função para carregar a ementa semanal a partir de um ficheiro
@param nomeficheiro  Nome do ficheiro
@param ementas  Array de ementas
@return int   ]Retorna o total de ementas carregadas*/
int carregarEmenta(char* nomeficheiro, Ementa ementas[]) {
    FILE* ficheiro = fopen(nomeficheiro, "r");
    if (ficheiro == NULL) {
        printf("Erro ao abrir ficheiro de ementa\n");
        return 0;
    }

    int totalEmentas = 0; 
    while (fscanf(ficheiro, "%9[^;];%11[^;];%99[^;];%d;%99[^;];%d;%99[^;];%d;%99[^;];%d\n", 
                    ementas[totalEmentas].diaSemana,
                    ementas[totalEmentas].data,
                    ementas[totalEmentas].pratoPeixe, &ementas[totalEmentas].caloriasPeixe,
                    ementas[totalEmentas].pratoCarne, &ementas[totalEmentas].caloriasCarne, 
                    ementas[totalEmentas].pratoDieta, &ementas[totalEmentas].caloriasDieta, 
                    ementas[totalEmentas].pratoVegetariano, &ementas[totalEmentas].caloriasVegetariano) == 10) {
                totalEmentas++;
        if (totalEmentas >= 5) {
            break;
        }
    }

    fclose(ficheiro);
    printf("Carregamento de ementa semanal foi realizado com sucesso!\n");
    return totalEmentas; 
}

\end{lstlisting}

\begin{lstlisting}[caption={carregarEscolhas}, label={lst:main}]
/**
@brief  Função para carregar as escolhas dos utentes a partir de um ficheiro
@param nomeFicheiro  Nome do ficheiro
@param escolhas  Array de escolhas
@return int  Retorna o total de escolhas carregadas*/
int carregarEscolhas(char* nomeFicheiro, Escolha escolhas[]) {
    FILE* ficheiro = fopen(nomeFicheiro, "r");
    if (ficheiro == NULL) {
        printf("Erro ao abrir o ficheiro!");
        return 0; 
    }

    int totalEscolhas = 0;
    while (fscanf(ficheiro, "%19[^;];%d;%19[^\n]\n", 
                  escolhas[totalEscolhas].diaSemana, 
                  &escolhas[totalEscolhas].numeroFuncionario, 
                  escolhas[totalEscolhas].tipoPrato) == 3) {
        totalEscolhas++;
        if (totalEscolhas >= MAX_ESCOLHAS) { // Limita ao número máximo de escolhas
            printf("Limite máximo de refeições atingido %d \n", MAX_ESCOLHAS);
            break;
        }
    }
    fclose(ficheiro);

    printf("Carregamento das escolhas realizado com sucesso!\n");
    return totalEscolhas;
}
\end{lstlisting}

\begin{lstlisting}[caption={listarRefeicoesDia}, label={lst:main}]
/**
@brief  Função para listar as refeições de um dia específico
@param escolhas  Array de escolhas
@param totalEscolhas  Total de escolhas
@param funcionarios  Array de funcionários
@param totalFuncionarios  Total de funcionários
@param diaSemana  Dia da semana*/
void listarRefeicoesDia(Escolha* escolhas, int totalEscolhas, Funcionario* funcionarios, int totalFuncionarios, const char* diaSemana) {
    printf("Refeições para o dia %s:\n", diaSemana);
    printf("----------------------------------------------------\n");
    printf("| Nº Funcionário | Nome            | Prato         |\n");
    printf("----------------------------------------------------\n");

    int encontrado = 0; // Flag para indicar se alguma refeição foi encontrada

    for (int i = 0; i < totalEscolhas; i++) {
        if (strcmp(escolhas[i].diaSemana, diaSemana) == 0) {
            // Encontra o nome do funcionário correspondente
            for (int j = 0; j < totalFuncionarios; j++) {
                if (funcionarios[j].numero == escolhas[i].numeroFuncionario) {
                    printf(" |%15d|%-15s|%-15s| \n",
                           funcionarios[j].numero,
                           funcionarios[j].nome,
                           escolhas[i].tipoPrato);
                    encontrado = 1; // Marca que encontramos pelo menos uma refeição
                    break; // Para de procurar o funcionário correspondente
                }
            }
        }
    }

    if (!encontrado) {
        printf("Nenhuma refeição foi registrada para o dia de %s.\n", diaSemana);
    }
}
\end{lstlisting}

\begin{lstlisting}[caption={listarUtentesOrdemDecrescente}, label={lst:main}]
/**
@brief Função para listar os utentes em ordem decrescente de número de funcionário
@param escolhas Array de escolhas
@param totalEscolhas Total de Escolhas Possivel
@param funcionarios  Array de funcionários
@param totalFuncionarios  Total de Funcionários*/
void listarUtentesOrdemDecrescente(Escolha* escolhas, int totalEscolhas, Funcionario* funcionarios, int totalFuncionarios) {
    // Arrays para armazenar refeições e despesas por funcionário
    int refeicoes[totalFuncionarios];
    float despesas[totalFuncionarios];

    // Inicializa os contadores
    for (int i = 0; i < totalFuncionarios; i++) {
        refeicoes[i] = 0;
        despesas[i] = 0.0f;
    }

    // Calcula o número de refeições e a despesa total para cada funcionário
    for (int i = 0; i < totalEscolhas; i++) {
        for (int j = 0; j < totalFuncionarios; j++) {
            if (escolhas[i].numeroFuncionario == funcionarios[j].numero) {
                refeicoes[j]++;
                despesas[j] += 6.00;
            }
        }
    }

    // Ordena os funcionários em ordem decrescente de número de funcionário
    for (int i = 0; i < totalFuncionarios - 1; i++) {
        for (int j = i + 1; j < totalFuncionarios; j++) {
            if (funcionarios[i].numero< funcionarios[j].numero) {
                // Troca os funcionários
                Funcionario tempFuncionario = funcionarios[i];
                funcionarios[i] = funcionarios[j];
                funcionarios[j] = tempFuncionario;

                // Troca os dados correspondentes
                int tempRefeicoes = refeicoes[i];
                refeicoes[i] = refeicoes[j];
                refeicoes[j] = tempRefeicoes;

                float tempDespesas = despesas[i];
                despesas[i] = despesas[j];
                despesas[j] = tempDespesas;
            }
        }
    }

    // Exibe os dados ordenados
    printf("Relatório de Utentes - Recursos Humanos\n");
    printf("-------------------------------------------------------------\n");
    printf("| Nº Funcionário | Nome            | Refeições | Total (€) |\n");
    printf("-------------------------------------------------------------\n");
    for (int i = 0; i < totalFuncionarios; i++) {
        printf("| %-15d | %-15s | %-10d | %-10.2f |\n",
               funcionarios[i].numero,
               funcionarios[i].nome,
               refeicoes[i],
               despesas[i]);
    }
}

\end{lstlisting}

\begin{lstlisting}[caption={listarRefeicoesUtente}, label={lst:main}]
/**
@brief  Função para listar as refeições e calorias de um utente durante um período
@param escolhas  Array de escolhas
@param totalEscolhas  Total de escolhas
@param ementas  Array de ementas
@param totalFuncionarios  Total de funcionários
@param totalEmentas  Total de ementas*/
void listarRefeicoesUtente(Escolha* escolhas, int totalEscolhas, Ementa* ementas, int totalFuncionarios, int totalEmentas) {
    int numeroFuncionario;
    printf("Escreva o número do funcionário: ");
    scanf("%d", &numeroFuncionario);

    // Verifica se o número do funcionário é válido antes de pedir os dias
    if (numeroFuncionario <= 0  numeroFuncionario > totalFuncionarios) {
        printf("Erro: Número de funcionário %d não existe.\n", numeroFuncionario);
        return;  // Sai da função e retorna ao menu
    }

    char DiaInicio[20], DiaFim[20];
    printf("Escreva o dia de início (ex: Segunda): ");
    scanf("%s", DiaInicio);
    printf("Escreva o dia de fim (ex: Sexta): ");
    scanf("%s", DiaFim);

    int numeroInicio = diaSemanaParaNumero(DiaInicio);
    int numeroFim = diaSemanaParaNumero(DiaFim);


    if (numeroInicio == -1  numeroFim == -1) {
        printf("Erro: Dia de início ou fim inválido.\n");
        return;
    }

    printf("Refeições e calorias do funcionário %d durante o período de %s a %s:\n", numeroFuncionario, DiaInicio, DiaFim);
    printf("---------------------------------------------------------\n");
    printf("|  Dia da Semana  |   Tipo de Prato    |   Calorias   |\n");
    printf("---------------------------------------------------------\n");

    int numeroRefeicoes = 0;
    int totalCalorias = 0;
    // Passa pelas escolhas e buscar correspondências na ementa
    for (int i = 0; i < totalEscolhas; i++) {
        if (escolhas[i].numeroFuncionario == numeroFuncionario) {
            for (int j = 0; j < totalEmentas; j++) {
                if (strcmp(escolhas[i].diaSemana, ementas[j].diaSemana) == 0) {
                    int indiceAtual = diaSemanaParaNumero(ementas[j].diaSemana);

                    if (estaNoIntervalo(indiceAtual, numeroInicio, numeroFim)) {
                        // Determinar as calorias do prato específico
                        int calorias = 0;

                        if (strcmp(escolhas[i].tipoPrato, "Peixe") == 0) {
                            calorias = ementas[j].caloriasPeixe;
                        } else if (strcmp(escolhas[i].tipoPrato, "Carne") == 0) {
                            calorias = ementas[j].caloriasCarne;
                        } else if (strcmp(escolhas[i].tipoPrato, "Dieta") == 0) {
                            calorias = ementas[j].caloriasDieta;
                        } else if (strcmp(escolhas[i].tipoPrato, "Vegetariano") == 0) {
                            calorias = ementas[j].caloriasVegetariano;
                        }

                        // Atualizar valores e exibir o prato específico
                        totalCalorias += calorias; // Acumular calorias
                        printf("| %16s | %15s | %5d |\n", ementas[j].diaSemana, escolhas[i].tipoPrato, calorias);
                        numeroRefeicoes++;
                    }
                }
            }
        }
    }

    printf("---------------------------------------------------------\n");
    if (numeroRefeicoes > 0) {
        printf("Total de refeições: %d\n", numeroRefeicoes);
        printf("Total de calorias consumidas no período: %d\n", totalCalorias);
    } else {
        printf("Nenhuma refeição registrada para o funcionário %d durante o período de %s a %s.\n", numeroFuncionario, DiaInicio, DiaFim);
    }
}
\end{lstlisting}
\begin{lstlisting}[caption={calcularMediaCaloriasEspaco}, label={lst:main}]
/**
@brief  Função para calcular a média de calorias consumidas no período
@param ementas  Array de ementas
@param totalEmentas  Total de ementas
@param escolhas  Array de escolhas
@param totalEscolhas  Total de escolhas
@param diainicio  Dia de início
@param diafim  Dia de fim*/
// Função para calcular a média de calorias consumidas no período
void calcularMediaCaloriasEspaco(Ementa* ementas, int totalEmentas, Escolha* escolhas, int totalEscolhas, char* diainicio, char* diafim) {
    // Converte os dias de início e fim em números
    int numeroDiaInicio = diaSemanaParaNumero(diainicio);
    int numeroDiaFim = diaSemanaParaNumero(diafim);

    if (numeroDiaInicio == -1 || numeroDiaFim == -1) {
        printf("Erro: Dia de início ou fim inválido.\n");
        return;
    }
    // Inicializa as variáveis para somar as calorias por dia
    int caloriasDia[5] = {0};  // 0: Segunda, 1: Terça, 2: Quarta, 3: Quinta, 4: Sexta
    int numeroRefeicoesPorDia[5] = {0};
    int CaloriasPrato = 0;

    // Percorre todas as escolhas de refeição
    for (int i = 0; i < totalEscolhas; i++) {
        // Obtém o número do dia da escolha
        int numeroDiaEscolha = diaSemanaParaNumero(escolhas[i].diaSemana);

        // Verifica se o dia da escolha está dentro do intervalo de dias
        if (numeroDiaEscolha >= numeroDiaInicio && numeroDiaEscolha <= numeroDiaFim) {
            // Percorre as ementas para verificar a correspondência dos pratos
            for (int y = 0; y < totalEmentas; y++) {
                if (numeroDiaEscolha == diaSemanaParaNumero(ementas[y].diaSemana)) {
                    // Verifica o tipo de prato e calcula as calorias
                    if (strcmp(escolhas[i].tipoPrato, "Peixe") == 0) {
                        CaloriasPrato = ementas[y].caloriasPeixe;
                    } else if (strcmp(escolhas[i].tipoPrato, "Carne") == 0) {
                        CaloriasPrato = ementas[y].caloriasCarne;
                    } else if (strcmp(escolhas[i].tipoPrato, "Dieta") == 0) {
                        CaloriasPrato = ementas[y].caloriasDieta;
                    } else if (strcmp(escolhas[i].tipoPrato, "Vegetariano") == 0) {
                        CaloriasPrato = ementas[y].caloriasVegetariano;
                    }

                    // Adiciona as calorias do prato escolhido ao total de calorias do dia correspondente
                    caloriasDia[numeroDiaEscolha] += CaloriasPrato;
                    numeroRefeicoesPorDia[numeroDiaEscolha]++;
                    break; // Interrompe o loop de ementas, pois encontramos o prato para o dia
                }
            }
        }
    }
    / Exibe as médias de calorias consumidas por refeição para cada dia da semana
    printf("Média de calorias consumidas por refeição de cada dia da semana no período de %s a %s:\n", diainicio, diafim);
    printf("-------------------------------------------------------------\n");

    // Array com os dias da semana
    const char* diasSemana[] = {"Segunda", "Terça", "Quarta", "Quinta", "Sexta"};

    for (int k = 0; k < 5; k++) {
        if (numeroRefeicoesPorDia[k] > 0) {
            float media = (float)caloriasDia[k] / numeroRefeicoesPorDia[k];
            printf("%s:  %d refeições, média: %.2f calorias por refeição.\n",
                diasSemana[k], numeroRefeicoesPorDia[k], media);
        }
    }
}
\end{lstlisting}

\begin{lstlisting}[caption={gerarTabelaEmentaUtente}, label={lst:main}]
/**
@brief  Função para gerar uma tabela de ementa para um utente
@param escolhas  Array de escolhas
@param totalEscolhas  Total de escolhas
@param ementas  Array de ementas
@param totalEmentas  Total de ementas
@param numeroFuncionario  Número do funcionário*/
void gerarTabelaEmentaUtente(Escolha* escolhas, int totalEscolhas, Ementa* ementas, int totalEmentas, int numeroFuncionario) {
    printf("| Dia Semana | Prato Escolhido | Calorias |\n");
    printf("==============================================\n");
    // Percorre todos os dias da semana para verificar se o funcionário fez escolha
    for (int i = 0; i < totalEmentas; i++) {
        char* refeicaoEscolhida = 0;
        int calorias = 0;

        // Verifica se o funcionário fez uma escolha para aquele dia
        for (int j = 0; j < totalEscolhas; j++) {
            if (escolhas[j].numeroFuncionario == numeroFuncionario && 
                strcmp(escolhas[j].diaSemana, ementas[i].diaSemana) == 0) {


                // Atribui as calorias conforme o prato escolhido
                if (strcmp(escolhas[j].tipoPrato, "Peixe") == 0) {
                    calorias = ementas[i].caloriasPeixe;
                } else if (strcmp(escolhas[j].tipoPrato, "Carne") == 0) {
                    calorias = ementas[i].caloriasCarne;
                } else if (strcmp(escolhas[j].tipoPrato, "Dieta") == 0) {
                    calorias = ementas[i].caloriasDieta;
                } else if (strcmp(escolhas[j].tipoPrato, "Vegetariano") == 0) {
                    calorias = ementas[i].caloriasVegetariano;
                }

                // Imprime os dados na tabela
                printf("| %15s | %15s | %-5d |\n", 
                       ementas[i].diaSemana, escolhas[j].tipoPrato , calorias);
                break;
            }
        }
    }
}
\end{lstlisting}
% Capítulo de Resultados
\chapter{Resultados}
A funcionalidade de mostrar o menu foi implementada corretamente, conforme esperado. Ao executar o programa, o menu é exibido de forma clara e interativa, permitindo ao utilizador selecionar diversas opções de acordo com o que é solicitado no enunciado do problema. As funções presentes no menu contemplam todas as funcionalidades exigidas pelo enunciado, incluindo a manipulação e processamento de dados, bem como a interação com o sistema de forma eficiente.

As opções do menu foram testadas com entradas válidas e inválidas, e o sistema comportou-se corretamente em todas as situações. O menu também apresenta mensagens informativas e de erro apropriadas, o que garante uma experiência para o utilizador intuitiva.

A imagem de prova que o menu é exibido corretamente encontra-se abaixo:

\includegraphics{Resultado_menu.jpg}

Após testar todas as operações presentes no menu (1-9) os resultados apresentam-se na tabela abaixo:
\begin{table}[H]
    \centering
    \caption{Resultados obtidos nos testes.}
    \label{tab:resultados_exemplo}
    \begin{tabular}{|c|c|c|}
        \hline
        Teste & Resultado Esperado & Resultado Obtido \\ \hline
        Teste 1 & Sucesso & Sucesso \\ \hline
        Teste 2 & Sucesso & Sucesso \\ \hline
        Teste 3 & Sucesso & Sucesso \\ \hline
        Teste 4 & Sucesso & Sucesso \\ \hline
        Teste 5 & Sucesso & Sucesso \\ \hline
        Teste 6 & Sucesso & Sucesso \\ \hline
        Teste 7 & Sucesso & Sucesso \\ \hline
        Teste 8 & Sucesso & Sucesso \\ \hline
        Teste 9 & Sucesso & Sucesso \\ \hline
    \end{tabular}
\end{table}

% Conclusão
\chapter{Conclusão}
O desenvolvimento deste trabalho permitiu consolidar competências em programação imperativa, gestão de dados e boas práticas de desenvolvimento de trabalho de grupo. A solução proposta responde de forma eficiente às necessidades de gestão do Espaço Social, o que permite o processamento de dados de funcionários, ementas e escolhas, e gerar relatórios úteis para a administração.
A estrutura do programa, com separação de responsabilidades em diferentes ficheiros e funções, facilita a manutenção e escalabilidade da aplicação. A utilização de técnicas como leitura e validação de ficheiros, organização de dados e cálculo de métricas reforçou a compreensão dos conceitos aprendidos ao longo da unidade curricular.
Por fim, o uso de ferramentas como Makefile, Git e Doxygen assegurou um desenvolvimento estruturado e documentado, alinhado às boas práticas de engenharia de software. Este trabalho destacou a importância da colaboração em equipa, da organização no desenvolvimento e da clareza na documentação para garantir o sucesso de projetos de programação.

\end{document}
=======
\documentclass[a4paper,12pt]{report}

% Pacotes necessários
\usepackage[utf8]{inputenc}
\usepackage[portuguese]{babel}
\usepackage{float}
\usepackage{amsmath}
\usepackage{listings}
\usepackage{hyperref}
\usepackage{tocbibind}
\usepackage{geometry}
\geometry{a4paper, margin=2.5cm}

% Configuração para listagens de código
\lstset{
    language=Python,
    basicstyle=\ttfamily\small,
    numbers=left,
    stepnumber=1,
    numbersep=5pt,
    showspaces=false,
    showstringspaces=false,
    showtabs=false,
    frame=single,
    breaklines=true,
    breakatwhitespace=true,
    tabsize=4
}

% Início do documento
\begin{document}

% Capa
\begin{titlepage}
    \centering
    {\Large Universidade Exemplo}\\[1.5cm]
    {\Large Relatório de Projeto}\\[2cm]
    {\Huge \textbf{Título do Trabalho}}\\[2cm]
    {\large Autores: Nome do Autor 1, Nome do Autor 2}\\[0.5cm]
    {\large Docente: Nome do Docente}\\[1.5cm]
    {\large \today}\\
\end{titlepage}

\tableofcontents
\newpage

% Introdução
\chapter{Introdução}
A introdução deve apresentar o contexto do trabalho, os objetivos principais e uma visão geral da estrutura do documento. Por exemplo, neste relatório serão discutidos os seguintes tópicos:

\begin{itemize}
    \item Motivação para o desenvolvimento do projeto.
    \item Problemas abordados.
    \item Soluções propostas.
\end{itemize}

% Capítulo de Desenvolvimento
\chapter{Desenvolvimento}

\section{Descrição do Sistema}
Nesta secção, será apresentado o conceito geral do sistema, incluindo:

\begin{description}
    \item[Arquitetura:] Descrição geral da arquitetura do programa, com destaque para os principais componentes.
    \item[Funcionalidades:] Resumo das funcionalidades implementadas.
\end{description}

\subsection{Detalhes de Implementação}
Segue abaixo um exemplo de código:

\begin{lstlisting}[caption={Função principal do programa}, label={lst:main}]
def main():
    print("Este é um exemplo de código.")
    # Mais detalhes de implementação aqui
\end{lstlisting}

% Capítulo de Resultados
\chapter{Resultados}
Nesta secção, apresentamos os resultados obtidos com a aplicação do sistema, organizados da seguinte forma:

\begin{enumerate}
    \item Testes realizados.
    \item Métricas de avaliação.
    \item Comparação com soluções existentes.
\end{enumerate}

Os resultados são apresentados na tabela abaixo:

\begin{table}[H]
    \centering
    \caption{Resultados obtidos nos testes.}
    \label{tab:resultados_exemplo}
        \begin{tabular}{|c|c|c|}
    \hline
        Teste & Resultado Esperado & Resultado Obtido \\ \hline
        Teste 1 & Sucesso & Sucesso \\ \hline
        Teste 2 & Sucesso & Falha \\ \hline
        \end{tabular}
    \end{table}


% Conclusão
\chapter{Conclusão}
A conclusão deve sumarizar os resultados alcançados, destacar as principais contribuições do trabalho e apresentar sugestões para trabalhos futuros.

% Referências
\bibliographystyle{plain}
\bibliography{bibliografia}

\end{document}
>>>>>>> 9230929953353090db214b8495156293b68fbc40
